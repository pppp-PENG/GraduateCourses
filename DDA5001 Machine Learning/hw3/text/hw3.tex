\documentclass[12pt, a4paper, oneside]{article}
\usepackage[T1]{fontenc}
\usepackage{amsmath, amsthm, amssymb, bm, booktabs, color, enumitem, float, graphicx, hyperref, mathrsfs, titling}
\usepackage[UTF8, scheme = plain]{ctex}
\usepackage{graphicx, minted, listings, subcaption}
\usepackage{grffile}
\title{\textbf{Homework 3}}
\setlength{\droptitle}{-10em}
\author{PENG Qiheng \\ Student ID\: 225040065}
\date{\today}
\linespread{1.5}
\newcounter{problemname}
\newenvironment{problem}{\stepcounter{problemname}\par\noindent{Problem \arabic{problemname}. }}{\par}
\newenvironment{solution}{\par\noindent{Solution. }}{\par}

\begin{document}

\maketitle

\begin{problem}
\begin{enumerate}[label = (a\arabic*)]
    \item \begin{align} 
        \notag X = \begin{bmatrix}
            1 & x_1 & x_1^2 & \cdots & x_1^8 \\
            1 & x_2 & x_2^2 & \cdots & x_2^8 \\
            \vdots & \vdots & \vdots & \vdots & \vdots \\
            1 & x_n & x_n^2 & \cdots & x_n^8
        \end{bmatrix}, \quad y = \begin{bmatrix}
            y_1 \\
            y_2 \\
            \vdots \\
            y_n
        \end{bmatrix}
    \end{align}
    \item \begin{align} \notag \hat \theta = \left [ \begin{matrix}
        0.60775979 \\
        -7.25862115 \\
        15.3450059 \\
        17.26526602 \\
        -46.39728808 \\
        -10.63103168 \\
        41.47282583 \\
        1.75929095 \\
        -11.22913232
    \end{matrix} \right ] \end{align}
    \begin{figure}[H]
        \centering
        \includegraphics[width=0.6\textwidth]{../code_release/p1/a2.png}
        \caption{Ridge regression with different $\lambda$}
    \end{figure}
    \item \begin{align} \notag {||\bf{X}_{test} \hat \theta - \bf{y}_{test}||}_2 = 4.5607 \end{align}
\end{enumerate}
\begin{enumerate}[label = (b\arabic*)]
    \item The figure is as follows:
    \begin{figure}[H]
        \centering
        \includegraphics[width=0.6\textwidth]{../code_release/p1/b1.png}
        \caption{Lasso regression with different $\lambda$}
    \end{figure}
    \item The figures are as follows:
    \begin{figure}[H]
        \centering
        \begin{subfigure}[b]{0.48\linewidth}
            \centering
            \includegraphics[width=\linewidth]{../code_release/p1/b2-0.01.png}
            \caption{$\lambda$ = 0.01}
        \end{subfigure}\hfill
        \begin{subfigure}[b]{0.48\linewidth}
            \centering
            \includegraphics[width=\linewidth]{../code_release/p1/b2-0.1.png}
            \caption{$\lambda$ = 0.1}
        \end{subfigure}
        \begin{subfigure}[b]{0.48\linewidth}
            \centering
            \includegraphics[width=\linewidth]{../code_release/p1/b2-0.8.png}
            \caption{$\lambda$ = 0.8}
        \end{subfigure}\hfill
        \begin{subfigure}[b]{0.48\linewidth}
            \centering
            \includegraphics[width=\linewidth]{../code_release/p1/b2-5.0.png}
            \caption{$\lambda$ = 5}
        \end{subfigure}
        \caption{Lasso regression coefficients with different $\lambda$}
    \end{figure}
    \item \begin{align}
        \notag \lambda = 0.01, \quad \text{test error} = 0.6634 \\
        \notag \lambda = 0.1, \quad \text{test error} = 0.6341 \\
        \notag \lambda = 0.8, \quad \text{test error} = 0.6493 \\
        \notag \lambda = 5, \quad \text{test error} = 0.8207
    \end{align}
\end{enumerate}
\end{problem}

% duplicate figure fragment removed

\newpage
\begin{problem}
\begin{enumerate}[label = (\arabic*)]
    \item  The top 40 eigenfaces are as follows:
    \begin{figure}[H]
        \centering
        \includegraphics[width=0.8\textwidth]{../code_release/p2/1.png}
        \caption{Top 40 eigenfaces}
    \end{figure}
    \item  The figure is as follows:
    \begin{figure}[H]
        \centering
        \includegraphics[width=0.8\textwidth]{../code_release/p2/2.png}
        \caption{Corresponding reconstructed images with k}
    \end{figure}
    As k increases, the reconstruction details significantly increase (facial contours and expression details gradually recover), and if k is too small, important structures will be lost and blurred; Excessive k can restore noise and individual differences, tending towards overfitting the training set.
    \item  The figure is as follows:
    \begin{figure}[H]
        \centering
        \includegraphics[width=0.8\textwidth]{../code_release/p2/3.png}
        \caption{SNR under different choices of k}
    \end{figure}
\end{enumerate}
\end{problem}

\newpage
\begin{problem}
\begin{enumerate}[label = (\alph*)]
    \item  The figures are as follows:
    \begin{figure}[H]
        \centering
        \begin{subfigure}[b]{0.48\linewidth}
            \centering
            \includegraphics[width=\linewidth]{../code_release/p3/plots/poly_deg1_valerr_vs_C.png}
            \caption{Polynomial degree = 1}
        \end{subfigure}\hfill
        \begin{subfigure}[b]{0.48\linewidth}
            \centering
            \includegraphics[width=\linewidth]{../code_release/p3/plots/poly_deg2_valerr_vs_C.png}
            \caption{Polynomial degree = 2}
        \end{subfigure}
        \caption{Validation error vs C for different polynomial degrees}
    \end{figure}
    poly\_deg1 best C = 46.4159, val\_err = 0.094 \\
    poly\_deg1 test error (retrained on whole training set) = 0.0441 \\
    poly\_deg2 best C = 10, val\_err = 0.092 \\
    poly\_deg2 test error (retrained on whole training set) = 0.0355
    \item  The figure is as follows:
    \begin{figure}[H]
        \centering
        \includegraphics[width=0.8\textwidth]{../code_release/p3/plots/rbf_valerr_vs_C_per_gamma.png}
        \caption{Validation error vs gamma and C for RBF kernel}
    \end{figure}
    RBF best gamma=0.01, best C=10 \\
    RBF test error (retrained on whole training set) = 0.0005
\end{enumerate}
\end{problem}

\end{document}