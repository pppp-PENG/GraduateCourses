\documentclass[12pt, a4paper, oneside]{article}
\usepackage[T1]{fontenc}
\usepackage{amsmath, amsthm, amssymb, bm, booktabs, color, enumitem, float, graphicx, hyperref, mathrsfs, titling}
\usepackage[UTF8, scheme = plain]{ctex}
\usepackage{graphicx, minted, listings, subfigure}
\usepackage{grffile}
\title{\textbf{Homework 1}}
\setlength{\droptitle}{-10em}
\author{PENG Qiheng \\ Student ID\: 225040065}
\date{\today}
\linespread{1.5}
\newcounter{problemname}
\newenvironment{problem}{\stepcounter{problemname}\par\noindent{Problem \arabic{problemname}. }}{\par}
\newenvironment{solution}{\par\noindent{Solution. }}{\par}

\begin{document}

\maketitle

\begin{problem}
\begin{enumerate}[label = (\alph*)]
    \item  The solving process is as follows:
    \begin{figure}[H]
        \centering
        \includegraphics[width=0.8\textwidth]{../code/p1_pic.jpg}
        \caption{Solving process for (a)}
    \end{figure}
    \item  Yes, 
    but one condition needs to be met: 
    the objective function coefficients are all integers. 
    Because if the coefficients of the objective function are all integers, 
    then the objective value of any integer solution must also be an integer, 
    so the optimal integer solution z must also be an integer. \\
    In this question, the objective function coefficients are 2, 3, 4, and 7, 
    all of which are integers, so this condition is satisfied.
\end{enumerate}
\end{problem}

\newpage
\begin{problem}
\begin{enumerate}[label = (\alph*)]
    \item  \begin{gather}
        \notag \nabla f(x) = \begin{bmatrix}
            4x_1^3 + 6x_1^2 - 4x_1 x_2 \\
            -2x_1^2 + 8x_2
        \end{bmatrix} \\
        \notag \text{Let } \nabla f(x) = \mathbf{0} \quad
        \Rightarrow \begin{cases}
            x_1 = 0 \\
            x_2 = 0
        \end{cases} \text{or } \begin{cases}
            x_1 = -2 \\
            x_2 = 1
        \end{cases} \\
        \notag \text{While } \nabla^2 f(x) = \begin{bmatrix}
            12x_1^2 + 12x_1 - 4x_2 & -4x_1 \\
            -4x_1 & 8
        \end{bmatrix} \\
        \notag \nabla^2 f(0,0) = \begin{bmatrix}
            0 & 0 \\
            0 & 8
        \end{bmatrix} , \quad \nabla^2
        f(-2,1) = \begin{bmatrix}
            20 & 8 \\
            8 & 8
        \end{bmatrix} \text{(PD)}
    \end{gather}
    Thus, $(0, 0)$ is a saddle point, while $(-2,1)$ is a local minimum.
    \item The contour plot is as follows:
    \begin{figure}[H]
        \centering
        \includegraphics[width=0.4\textwidth]{../code/p2.png}
        \caption{Contour plot of $f(x)$}
    \end{figure}
    From the contour map, 
    it can be seen that the function presents a concave shape near the point $(-2, 1)$
    and is the only local minimum point. \\
    When $x_1 \rightarrow \infty or x_2 \rightarrow \infty$, $f(x) \rightarrow +\infty$,
    so there is a global minimum at $(-2, 1)$.
\end{enumerate}
\end{problem}

\newpage
\begin{problem}
\begin{enumerate}[label = (\alph*)]
    \item  Using the Leading Principle Minors method: \\
    $A_1$ is indefinite; \\
    $A_2$ is positive semidefinite; \\
    $A_3$ is positive definite; \\
    $A_4$ is indefinite. \\
    Calculating the eigenvalues of $A_3$ and $A_4$: \\
    \begin{gather}
        \notag |A_3 - \lambda I| = \begin{vmatrix}
            1 - \lambda & 0 & 1 \\
            0 & 1 - \lambda & -1 \\
            0 & 2 & 4 - \lambda
        \end{vmatrix} = (1 - \lambda)(2 - \lambda)(3 - \lambda) = 0 \\
        \notag \Rightarrow \lambda_1 = 1, \lambda_2 = 2, \lambda_3 = 3 \\
        \notag |A_4 - \lambda I| = \begin{vmatrix}
            1 - \lambda & 0 & 1 \\
            0 & -1 - \lambda & -1 \\
            -1 & 1 & - \lambda \\
        \end{vmatrix} = - \lambda(\lambda^2 + 1) = 0 \\
        \notag \Rightarrow \lambda_1 = 0, \lambda_2 = i, \lambda_3 = -i
    \end{gather}
\end{enumerate}
\end{problem}

\end{document}
