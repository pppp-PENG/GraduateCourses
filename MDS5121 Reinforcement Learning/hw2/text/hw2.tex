\documentclass[12pt, a4paper, oneside]{article}
\usepackage[T1]{fontenc}
\usepackage{amsmath, amsthm, amssymb, bm, booktabs, color, enumitem, float, graphicx, hyperref, mathrsfs, titling}
\usepackage[UTF8, scheme = plain]{ctex}
\usepackage{graphicx, minted, listings, subfigure}
\usepackage{grffile}
\title{\textbf{Homework 2}}
\setlength{\droptitle}{-10em}
\author{PENG Qiheng \\ Student ID\: 225040065}
\date{\today}
\linespread{1.5}
\newcounter{problemname}
\newenvironment{problem}{\stepcounter{problemname}\par\noindent{Problem \arabic{problemname}. }}{\par}
\newenvironment{solution}{\par\noindent{Solution. }}{\par}

\begin{document}

\maketitle

\begin{problem}
\begin{enumerate}
    \item \begin{align}
        \notag \text{state space} \quad & \mathcal{S} = \{0, 1, \ldots, n\} \\
        \notag \text{action space} \quad & \mathcal{A} = \{'c', 'r'\} \\
        \notag \text{transition function} \quad & P(s'|s, a) = \begin{cases}
            \frac{n-i}{n}, & a = 'c', s' = s + 1 \\
            \frac{i}{n}, & a = 'c', s' = s - 1 \\
            \frac{1}{n}, & a = 'r', \forall s' \in \mathcal{S} \\
            0, & \text{otherwise}
        \end{cases} \\
        \notag \text{reward function} \quad & R(s, a, s') = \begin{cases}
            1, & s' = n \\
            0, & \text{otherwise}
        \end{cases} \\
        \notag \text{optimal value function} \quad & V^*(s) = \\
        \notag \max_{a \in \mathcal{A}} \sum_{s' \in \mathcal{S}} & P(s' | s, a) \left( R(s, a, s') + \gamma V^*(s') \right)
    \end{align}
    \item  When $n = 5, \gamma = 1$, the optimal value function and policy are:
    \begin{align}
        \notag V^* & = [0.0, 1.2857, 1.3929, 1.4643, 1.5714, 1.0] \\
        \notag \pi^* & = ['-', 'r', 'c', 'c', 'c', '-']
    \end{align}
\end{enumerate}
\end{problem}

\newpage
\begin{problem}
    \begin{align}
        \notag V^* = R + \gamma P V^* \\
        \notag (I - \gamma P) V^* = R \\
        \notag V^* = {(I - \gamma P)}^{-1} R = \hat V = \Phi w \\
        \notag R = (I - \gamma P) \Phi w
    \end{align}
    So we need $\Phi$ to satisfy the following equation:
    \begin{align}
        \notag R = (I - \gamma P) \Phi w
    \end{align}
    Thus we can guarantees that $V^* = \Phi w$.
\end{problem}

\newpage
\begin{problem}
\begin{enumerate}
    \item  Iteration 1: \begin{align}
        \notag V_1(s_1) & = \max\{R(s_1, a) + \gamma \sum_{s' \in \mathcal{S}} P(s_1, a, s') V_0(s')\} \\
        \notag & = \max\{8 + 2.6, 10 + 1.2\} = 11.2 \\
        \notag V_1(s_2) & = \max\{R(s_2, a) + \gamma \sum_{s' \in \mathcal{S}} P(s_2, a, s') V_0(s')\} \\
        \notag & = \max\{1 + 3.3, -1 + 5.3\} = 4.3 \\
        \notag V_1(s_3) & = \max\{R(s_3, a) + \gamma \sum_{s' \in \mathcal{S}} P(s_3, a, s') V_0(s')\} \\
        \notag & = \max\{0, 0\} = 0
    \end{align}
    Iteration 2: \begin{align}
        \notag V_2(s_1) & = \max\{R(s_1, a) + \gamma \sum_{s' \in \mathcal{S}} P(s_1, a, s') V_1(s')\} \\
        \notag & = \max\{8 + 4.82, 10 + 1.55\} = 12.82 \\
        \notag V_2(s_2) & = \max\{R(s_2, a) + \gamma \sum_{s' \in \mathcal{S}} P(s_2, a, s') V_1(s')\} \\
        \notag & = \max\{1 + 4.65, -1 + 6.89\} = 5.89 \\
        \notag V_2(s_3) & = \max\{R(s_3, a) + \gamma \sum_{s' \in \mathcal{S}} P(s_3, a, s') V_1(s')\} \\
        \notag & = \max\{0, 0\} = 0
    \end{align}
    Thus the value function after two iterations is:
    \begin{align}
        \notag V_2(s_1) & = 12.82 \\
        \notag V_2(s_2) & = 5.89 \\
        \notag V_2(s_3) & = 0
    \end{align}
    \item  According to the calculation in part (a), we have:
    \begin{align}
        \notag \forall k \ge 2, \quad & V_k(s_i) \ge V_{2}(s_i) \\
        \notag \Rightarrow & V^*(s_1) \ge 12.82, \quad V^*(s_2) \ge 5.89, \quad V^*(s_3) = 0
    \end{align}
    Thus we can get: 
    \begin{align}
        \notag Q_{k+1}(s_1, a_1) - Q_{k+1}(s_1, a_2) & = -2 + 0.1 V_k(s_1) - 0.4 V_k(s_2) \ge 0 \\
        \notag Q_{k+1}(s_2, a_1) - Q_{k+1}(s_2, a_2) & = 2 - 0.2 V_k(s_1) \le 0 \\
        \notag Q_{k+1}(s_3, a_1) = Q_{k+1}(s_3, a_2) & = 0 \\
        \notag \forall k \ge 2
    \end{align}
    And we let $\pi_k(s_3) = a_1$, then the policy is:
    \begin{align}
        \notag \forall k \ge 2, \quad & \pi_k(s_1) = a_1, \quad \pi_k(s_2) = a_2, \quad \pi_k(s_3) = a_1
    \end{align}
    Hence the policy will not change anymore after iteration 2, i.e. $\pi_k(s) = \pi_2(s), \forall k \ge 2$.
    \item  \textbf{Key difference:} \\
    Value iteration directly iterate the value function, 
    update the value through the Bellman optimal equation until convergence. 
    Then extract the optimal strategy from the final value function. 
    It does not explicitly maintain policies, 
    and every update requires traversal of all actions. \\
    Policy iteration alternates 
    between policy evaluation and policy improvement. 
    The policy evaluation stage calculates the value function of the current strategy 
    (possibly iterating to convergence), 
    and the policy improvement stage updates the strategy based on the current value function. 
    It explicitly maintains the strategy and usually converges faster. \\
    \textbf{Applicable scenarios:} \\
    Value iteration is more suitable for situations with large state spaces and limited computing resources, 
    as it is simple to implement but may converge slowly (especially when γ approaches 1). 
    When no intermediate strategy is needed, value iteration is more appropriate. \\
    Policy iteration is more suitable for situations where the state space is medium and requires rapid convergence, 
    as the policy improvement steps may reach the optimal strategy faster. 
    When intermediate strategies such as online learning are needed, policy iteration is more optimal.
    \item  Unsure. \\
    Usually, it cannot be guaranteed to obtain the optimal strategy. 
    Due to insufficient policy evaluation, 
    value function errors may propagate to policy improvements, 
    which may not satisfy the Bellman optimal equation. 
    However, in some cases (such as MDP having a simple structure and a good initial strategy), 
    it may accidentally converge.
\end{enumerate}
\end{problem}


\newpage
\begin{problem}
    Here is the code link: \url{https://colab.research.google.com/drive/1C9rgDx-I_2YyvPmMarWNHYNrQiNH9o1V?usp=sharing} \\
    With using 5 different random seeds, Value Iteration achieve an average score $0.7366 \pm 0.0102$ 
    and Policy Iteration methods achieve an average score $0.7356 \pm 0.0059$. \\
    The output of the code is as follows:
    \begin{verbatim}
        Testing with 5 different random seeds
        ==================================================
        --- Testing with seed 666 ---
        Best score (Value Iterator) = 0.73.
        Best score (Policy Iterator) = 0.73.
        --- Testing with seed 667 ---
        Best score (Value Iterator) = 0.76.
        Best score (Policy Iterator) = 0.74.
        --- Testing with seed 668 ---
        Best score (Value Iterator) = 0.73.
        Best score (Policy Iterator) = 0.74.
        --- Testing with seed 669 ---
        Best score (Value Iterator) = 0.73.
        Best score (Policy Iterator) = 0.73.
        --- Testing with seed 670 ---
        Best score (Value Iterator) = 0.74.
        Best score (Policy Iterator) = 0.74.
    \end{verbatim}
\end{problem}

\end{document}